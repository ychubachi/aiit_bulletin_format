% maketitleすると,footerの下部がずれてしまう現象について.
% 12Q ok, 13Q ok, 14Q ok
% 9pt NG, 10pt ok, 12pt ok, 13pt ok, 14pt NG, 17pt以上 未調査
\documentclass[a4j, 12Q, twocolumn, twoside, draft]{jsarticle}
\usepackage{aiitbulletin}

%% 必要に応じてパッケージやマクロをここに追加してください
%\usepackage[dvipdfmx]{graphicx} % 図の取り込み(PDF対応)用
\usepackage{url}
\newcommand{\cmd}[1]{\texttt{\symbol{"5C}#1}} %"

\title{\LaTeX の新クラスファイル\texttt{jsarticle}を拡張した\\
論文向けスタイルパッケージ
}
\author{中鉢 欣秀 \thanks{産業技術大学院大学/Advanced Institute of
Industrial Technology} 産技 花子
\thanks{公立大学法人首都大学東京/Tokyo Metropolitan University}}
\etitle{English Title}
\eauthor{Yoshihide Chubachi \thanksmark[1] Hanako Sangi \thanksmark[2]}
\begin{abstract}
 Lorem ipsum dolor sit amet, consectetur adipiscing elit. Duis
 molestie ligula a diam hendrerit, id vulputate est
 feugiat. Suspendisse auctor neque a enim scelerisque, nec fermentum
 diam sodales. Pellentesque habitant morbi tristique senectus et netus
 et malesuada fames ac turpis egestas. Phasellus eu sodales neque. In
 interdum nisi at egestas aliquet. Phasellus vitae ullamcorper lectus,
 a commodo orci. Curabitur luctus viverra auctor. 
\end{abstract}
\keywords{AIIT, bulletin, 2013}


\begin{document}

%==========================
\maketitle
\section{はじめに}
 この文書では\LaTeX の新クラスファイルを基に作成したスタイルファイルに
 ついて述べる.日本語のレイアウトに欠かせない「行取り」のマクロを追加し,
 二段組の左右の行が揃うことをより徹底した.また,英文の表題,著者
 名,概要,キーワードを出力できるようにした.標準的なクラスファイルに最
 低限の拡張を施し,より美しく,実用的なレイアウトを実現することを狙う.
\section{スタイルの定義}
\subsection{基本版面}
A4用紙縦.
文字12Q(8.5pt)
二段組,1行27文字,段間2文字($27 \times 2 + 2 = 56$文字).
行数47行,行間75\%(行送り175\%)
.

余白は天(上部)30mm,地(22.5mm),ノド(内側)24mm,小口(18mm)をとる
こと.
製本時,余白には表題やページ番号が入る.

\subsection{\LaTeX における設定}

\texttt{geometory}パッケージで設定.
高さの補正\footnote{補正により,脚注を用いても
 \underline{行の下側が左右の段で揃う.}}.

\subsection{英文表題}

\cmd{etitle},\cmd{eauthor},\cmd{keywords}を追加した.

著者名に所属を記載する場合は\cmd{thanks}を使う.また,英文の著者名で所
属を参照するための記号を出力するために\cmd{thanksmark} コマンドを定義し
た.

\subsection{行取りマクロの利用方法}
左右の段の行を揃えるために,行取りをするマクロ\cmd{nlines}を作成した.
これを数式に対して使用すると次の通りになる.

\nlines{
  \begin{eqnarray}
    a_0=\frac{1}{\pi}\left[\int_0^{\pi} a \sin kt \cdot dt + \int_{\pi}^{2\pi} (-a) \sin kt \cdot dt\right]=0
  \end{eqnarray} 
}

このように,数式の上下の空白を調整し,左右の段で行が揃うようにする.

%\nlines{\LARGE 大きなフォントを設定しても大丈夫.自動的に3行取りになる.}

\section{おわりに}
この文書で述べた以外の図や表の挿入などは\texttt{jsarticle}の機能をその
まま利用することができる.詳しい使い方は・・・を参照すること.

\section{謝辞}
\cmd{nlines}を実装するにあたり
\url{http://www.dab.hi-ho.ne.jp/t-wata/tex/multicol.html}の記事を参考に
した.


寿限無寿限無五劫の摺り切れ海砂利水魚の水行末雲来末風来末.食う寝る所に
住む所藪柑子ブラコウジ.パイポパイポパイポのシューリンガングーリンダイ
のポンポコピーのポンポコナーの長久命の長助.

寿限無寿限無五劫の摺り切れ海砂利水魚の水行末雲来末風来末.食う寝る所に
住む所藪柑子ブラコウジ.パイポパイポパイポのシューリンガングーリンダイ
のポンポコピーのポンポコナーの長久命の長助.

\underline{寿限無寿限無五劫の摺り切れ}海砂利水魚の水行末雲来末風来末.食う寝る所に
住む所藪柑子ブラコウジ.パイポパイポパイポのシューリンガングーリンダイ
のポンポコピーのポンポコナーの長久命の長助.


\clearpage
\appendix
\section{文字数・行数の確認}
\subsection{横27文字}
\noindent
1234567890
1234567890
1234567890

寿限無寿限無五劫の摺り切れ海砂利水魚の水行末雲来末風来末.食う寝る所に
住む所藪柑子ブラコウジ.パイポパイポパイポのシューリンガングーリンダイ
のポンポコピーのポンポコナーの長久命の長助.

\subsection{左右の段の行}

% 寿限無寿限無五劫の摺り切れ海砂利水魚の水行末雲来末風来末.食う寝る所に
% 住む所藪柑子ブラコウジ.パイポパイポパイポのシューリンガングーリンダイ
% のポンポコピーのポンポコナーの長久命の長助.

寿限無寿限無五劫の摺り切れ海砂利水魚の水行末雲来末風来末.食う寝る所に
住む所藪柑子ブラコウジ.パイポパイポパイポの\underline{シューリンガングーリンダイ}


\section{ははは}
ああああ

寿限無寿限無五劫の摺り切れ海砂利水魚の水行末雲来末風来末.食う寝る所に
住む所藪柑子ブラコウジ.パイポパイポパイポのシューリンガングーリンダイ
のポンポコピーのポンポコナーの長久命の長助.

寿限無寿限無五劫の摺り切れ海砂利水魚の水行末雲来末風来末.食う寝る所に
住む所藪柑子ブラコウジ.パイポパイポパイポのシューリンガングーリンダイ
のポンポコピーのポンポコナーの長久命の長助.

寿限無寿限無五劫の摺り切れ海砂利水魚の水行末雲来末風来末.食う寝る所に
住む所藪柑子ブラコウジ.パイポパイポパイポのシューリンガングーリンダイ
のポンポコピーのポンポコナーの長久命の長助.

寿限無寿限無五劫の摺り切れ海砂利水魚の水行末雲来末風来末.食う寝る所に
住む所藪柑子ブラコウジ.パイポパイポパイポのシューリンガングーリンダイ
のポンポコピーのポンポコナーの長久命の長助.

寿限無寿限無五劫の摺り切れ海砂利水魚の水行末雲来末風来末.食う寝る所に
住む所藪柑子ブラコウジ.パイポパイポパイポのシューリンガングーリンダイ
のポンポコピーのポンポコナーの長久命の長助.

ここまで\footnote{当然,ここも\underline{左右が揃う}.}.

\newpage
\noindent 1\par \noindent 2\par \noindent 3\par \noindent 4\par \noindent 5\par
\noindent 6\par \noindent 7\par \noindent 8\par \noindent 9\par \noindent 0\par
\noindent 11\par \noindent 2\par \noindent 3\par \noindent 4\par \noindent 5\par
\noindent 6\par \noindent 7\par \noindent 8\par \noindent 9\par \noindent 0\par
\noindent 21\par \noindent 2\par \noindent 3\par \noindent 4\par \noindent 5\par
\noindent 6\par \noindent 7\par \noindent 8\par \noindent 9\par \noindent 0\par
\noindent 31\par \noindent 2\par \noindent 3\par \noindent 4\par \noindent 5\par
\noindent 6\par \noindent 7\par \noindent 8\par \noindent 9\par \noindent 0\par
\noindent 41\par \noindent 2\par \noindent 3\par \noindent 4\par \noindent 5\par
\noindent 6\par \noindent 7\underline{□□□□□} %\par \noindent 8\par \noindent 9\par \noindent 0\par
%\noindent 1\par %\noindent 2\par \noindent 3\par \noindent 4\par \noindent 5\par

\end{document}