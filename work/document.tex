\documentclass[a4j, 12Q, twocolumn, twoside]{jsarticle}
% 余白の設定
\usepackage{calc} % 数値演算を可能にするパッケージを利用する
\usepackage{okumacro}
\usepackage[
  top=30mm,          % 天(上余白)
  bottom=22.5truemm, % 地(下余白)
  inner=24mm,        % ノド(内側余白)
  outer=18truemm,    % 小口(外側余白)
  dvipdfm]{geometry}
% 行間の設定
%\renewcommand{\baselinestretch}{0.8670} % 二分あきにする場合

\title{\LaTeX の\texttt{jsarticle}スタイルを用いた印刷仕様の設定例}
\author{中鉢欣秀\thanks{産業技術大学院大学}}
\begin{abstract}
  \LaTeX のスタイルである\texttt{jsarticle}クラスを用いA4縦の用紙に文字の大きさ12Q(ほぼ8.5pt),
  \texttt{twocolumn},\texttt{twoside}でレイアウトを設定する例.
  紙面の余白を天30mm,ノド24mm,小口18mmとし,地を20mm程度確保する版面の大きさを計算した.
  その結果,行数48行ならば地を19.80mm,47行とするならば地を24.99mmとするのが最も適切な設定であることがわかった.
\end{abstract}

\pagestyle{empty}
\begin{document}
\maketitle

\section{はじめに}
  理系の論文の原稿は伝統的に\LaTeX で版組することが多いが,近年ではMS
  Wordで作成する機会も増えてきた.論文集などでは両者で共通のフォーマッ
  トを用意して,便宜をはかるところも多い.

  \LaTeX にしてもWordにしても,しっかりとしたフォーマットを0から作成す
  るのは大変である.\LaTeX のクラスファイルを作るのは\TeX の深い理解が
  必要であり,Word でテンプレートファイルを作成するにはある種の職人技が
  求められる.

  そこで,できるだけ少ない手間で両者共通のフォーマットを作成する方法を
  探ることにする.方針は\LaTeX の新しい標準的なクラスファイルである
  \texttt{jsarticle}クラスを基本とし,他に必要なパラメータを設定する.
  Wordはこれにあうように設定する.

\section{文字サイズ}

  論文の原稿では,欧文の文字の大きさを9ptとすることが多い.\LaTeX の
  jsarticle クラスファイルで9ptオプションを指定すると,欧文フォントは
  9ptとなるが,和文のフォントのサイズは欧文の文字に合わせるため約8.4pt
  となる.

  これにあわせてMS Wordでも同じレイアウトを得ようとしよう.ところが,
  Wordでは英文と和文で異なるフォントサイズを設定することができない.加
  えて,フォントのサイズとして8.4ptは指定できず,近い値として8.5ptを用
  いざるを得ない.このまま\LaTeX にあわせて行数を指定すると,和文主体の
  文章では少しばかり行間が詰まってしまう.

  ところで,jsarticleの12Qオプションは,計算するとフォントのサイズがほ
  ぼ8.5ptとなる.そこで,\ruby{両者}{りょうしゃ}でフォントの大きさを揃
  えるならば,\LaTeX では12Q,Wordでは8.5ptを設定するのが得策である.
  jsarticleの12Qオプションはあまり広く認知されていないが,Wordと共通の
  版面を作るにはうってつけである.

  そこで,以下,文字の大きさは12Qとして,基本版面を設計していこう.

\section{長さの単位について}
  \LaTeX における$1 pt = \frac{1}{72.27} inch$である.また,$1 inch =
  25.4 mm$である.また,1Q = 1H = 0.25mmである.横方向にはQを,縦方向に
  はHを用いる.
\section{判型・左右の余白と字詰と字間}

  判型は一般的なA4サイズで見開き印刷とする.段組は2段組で段間は2字とし,
  字間はベタ(文字と文字の間をあけない)とする.

  いま,各段の字詰を27字にすると,1行あたり
  \[
   27字 + 2字 + 27字 = 56字
  \]
  となる.

  文字の大きさ12Qであるから,
  \[
   12Q \times 0.25 mm \times 56字 = 168 mm
  \]
  が版面の横幅である.A4の横は$210mm$であるから,左右の余白をノド(内側)
  24mm,小口(外側)18mmとすると,
  \[
   168mm + 24mm + 18mm = 210mm
  \]
  となり,ぴたりと収まる.

\section{行間}
  行間は二分四分(文字の大きさの75\%)とする.これは\texttt{jsarticle}
  の標準の行間がほぼ二分四分であるからだが,論文で一般的な二分(50\%)
  よりもわりとゆったりとしたものになる.\ruby{ルビ}{るび}や\kenten{圏
  点}を入れても問題なく,タブレットなど電子媒体で読む場合でも読みやすく
  なるものと思われる.

  行間を75\%とし文字の高さと行間とを合わせた行送り(175\%)を計算すると,
  \[
   12H \times 1.75 = 21H = 5.25 mm
  \]
  となる.

  いま,行数を47行とすると
  \[
   5.25 mm \times 46 + 3mm = 244.5mm
  \]
  が版面の高さとなる.A4縦は297mmであるから,
  \[
   297mm - 244.5mm = 52.5mm
  \]
  が余白となる.柱を入れるために天(上余白)を30mmとると,地(下余白)
  は22.5mmとなる.

  実際には,\texttt{jsarticle}クラスで12Qオプションを設定した場合の行送り
  は\the\baselineskip であり,拡大率は$\frac{\the\mag}{1000}$であるから,
  \[
    16 \times \frac{1}{72.27} inch \times 25.4 mm/inch \times
    \frac{923}{1000} \approx 5.190 mm
  \]
  となる.これは,二分四分ほぼ近いものの,僅かな誤差がある.行数を47行
  として計算すると
  \[
   5.190 mm \times 46 + 3mm = 241.74mm
  \]

  Hで計算した場合は244.5mmであったから,比べると2.76mmほど短くなる.し
  かし,実用上問題はない.
  
\section{実際の設定}

  以上の考察から,実際の設定は次のとおりとした.

{\small
\begin{verbatim}
\documentclass[a4j,12Q,twocolumn,twoside]{jsarticle}
\usepackage[
  top=30mm,          % 天(上余白)
  bottom=22.5truemm, % 地(下余白)
  inner=24mm,        % ノド(内側余白)
  outer=18truemm,    % 小口(外側余白)
  dvipdfm]{geometry}
\end{verbatim}
}

\section{二分あきの設定}
  baselinestretch に倍率を設定し,行間を二分あきにすることを検討する.
  
  二分あきの場合,
  \[ 12H + 12H/2 \times 0.25mm/H= 4.5mm\]であるから,
  行送りを現在の
  $4.5mm \div 5.190mm \approx 0.867$
  倍すればよい.これを設定するには次の通りプリアンブルに記述する.

{\small
\begin{verbatim}
\renewcommand{\baselinestretch}{0.8670} % 二分あきにする場合
\end{verbatim}
}

  ただし,実際にこの設定を行ったところ,節の見出しなどで大きい文字を使
  うと左右の段が揃わない問題が発生する.
\section{TODO}
\begin{itemize}
 \item 表題などのフォントの大きさ・設定
\end{itemize}
\section{文字数・行数の確認}
\noindent
1234567890
1234567890
1234567890\footnote{寿限無寿限無五劫の摺り切れ海砂利水魚の水行末雲来末風来末.食う寝る所に住む所藪柑子ブラコウジ.パイポパイポパイポのシューリンガングーリンダイのポンポコピーのポンポコナーの長久命の長助.寿限無寿限無五劫の摺り切れ海砂利水魚の水行末雲来末風来末.}
\newpage
\noindent 1\par \noindent 2\par \noindent 3\par \noindent 4\par \noindent 5\par
\noindent 6\par \noindent 7\par \noindent 8\par \noindent 9\par \noindent 0\par
\noindent 1\par \noindent 2\par \noindent 3\par \noindent 4\par \noindent 5\par
\noindent 6\par \noindent 7\par \noindent 8\par \noindent 9\par \noindent 0\par
\noindent 1\par \noindent 2\par \noindent 3\par \noindent 4\par \noindent 5\par
\noindent 6\par \noindent 7\par \noindent 8\par \noindent 9\par \noindent 0\par
\noindent 1\par \noindent 2\par \noindent 3\par \noindent 4\par \noindent 5\par
\noindent 6\par \noindent 7\par \noindent 8\par \noindent 9\par \noindent 0\par
\noindent 1\par \noindent 2\par \noindent 3\par \noindent 4\par \noindent 5\par
\noindent 6\par \noindent 7\par \noindent 8\par \noindent 9\par \noindent 0\par
%\noindent 1\par %\noindent 2\par \noindent 3\par \noindent 4\par \noindent 5\par

% 文字サイズ・行間の詳細な計算
% ----------------------

% jsarticleの標準である10ptオプションでは,文字のサイズが13Q(9.21bp)になるように調整されます.
% 9ptオプションを設定すると行送りは15ptになります.15pt * 1/7.27インチ * 25.4mm / 0.25mm = 約21.085Hです.

% jsarticleのオプションで9ptの設定をすると,内部では10ptで版組し,これを913/1000倍することで求める出力を得ます.

% よって,文字のサイズは13Qに913/1000をかけて11.869Qとなります(これは約8.4bpです).
% 行送りは約21.085Hに913/1000をかけて19.251Hとなりますから,行間は19.251-11.869=7.382Hとなります.

% これをInDesignで再現したところ,一致しました.

% Wordを利用した設定
% ---------------

% このjsarticleで作製したスタイルにほぼ一致するように,Wordのスタイルを設定しました.

% Wordではフォントサイズを8.5bpとしています(8.4bpが設定できなかったため).

% 今後の作業
% --------

% LibreOfficeのWriterへの対応を予定しています.

% 12Q=8.5039ptであるから,jsarticleの12Qオプションを使用したほうがよいかもしれない.
\end{document}

