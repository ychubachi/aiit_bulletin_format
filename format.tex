% maketitleすると,footerの下部がずれてしまう.
% 10pt ok, 12pt ok, 13pt ok, 14pt NG, 
% 

\documentclass[a4j, twocolumn, twoside, draft]{jsarticle}

\usepackage{aiitbulletin}


\title{パイポパイポパイポのシューリンガングーリンダイの
ポンポコピーのポンポコナーの長久命の長
}
\author{著者 \thanks{AIIT}, 著者 \thanks{TMU}}
\etitle{English Title}
\eauthor{English Author}
\keywords{AIIT, bulletin, 2013}
\begin{abstract}
寿限無寿限無五劫の摺り切れ海砂利水魚の水
行末雲来末風来末.食う寝る所に住む所藪柑子ブラコウジ.パイポパイポパイ
ポのシューリンガングーリンダイのポンポコピーのポンポコナーの長久命の長
助.
\end{abstract}
\begin{document}

%==========================
%\maketitle
\section{はじめに}
\subsection{版面}
A4用紙縦.
文字12Q(8.5pt)
二段組,1行27文字,段間2文字($27 \times 2 + 2 = 56$文字).
行数47行,行間75\%(行送り175\%)\footnote{版面の幅は行末雲来末風来末.食う寝る所に住む所藪柑子ブラコウジ.パイポパイポパイ
ポのシューリンガングーリンダイのポンポコピーのポンポコナーの\underline{長久命の長}
}.


\section{つぎに}

余白は天(上部)30mm,地(22.5mm),ノド(内側)24mm,小口(18mm)をとる
こと.
製本時,余白には表題やページ番号が入る.

%\linespace{3} {\Large ああああああああああああああああああああああああああ}

寿限無寿限無五劫の摺り切れ海砂利水魚の水行末雲来末風来末.食う寝る所に
住む所藪柑子ブラコウジ.

パイポパイポパイポのシューリンガングーリンダイ
のポンポコピーのポンポコナーの長久命の長助.

寿限無寿限無五劫の摺り切れ海砂利水魚の水行末雲来末風来末.食う寝る所に
住む所藪柑子ブラコウジ.パイポパイポパイポのシューリンガングーリンダイ
のポンポコピーのポンポコナーの長久命の長助.

寿限無寿限無五劫の摺り切れ海砂利水魚の水行末雲来末風来末.食う寝る所に
住む所藪柑子ブラコウジ.パイポパイポパイポのシューリンガングーリンダイ
のポンポコピーのポンポコナーの長久命の長助.

寿限無寿限無五劫の摺り切れ海砂利水魚の水行末雲来末風来末.食う寝る所に
住む所藪柑子ブラコウジ.パイポパイポパイポのシューリンガングーリンダイ
のポンポコピーのポンポコナーの長久命の長助.

基本的に,左右の段の行は揃える.しかし,

% \linespace{3}{
% \begin{eqnarray}
% a_0=\frac{1}{\pi}\left[\int_0^{\pi} a \sin kt \cdot dt + \int_{\pi}^{2\pi} (-a) \sin kt \cdot dt\right]=0
% \end{eqnarray} 
% }

% 数式を入れるとずれてしまう.

寿限無寿限無五劫の摺り切れ海砂利水魚の水行末雲来末風来末.食う寝る所に
住む所藪柑子ブラコウジ.パイポパイポパイポのシューリンガングーリンダイ
のポンポコピーのポンポコナーの長久命の長助.

寿限無寿限無五劫の摺り切れ海砂利水魚の水行末雲来末風来末.\underline{食う寝る所に
住む所}藪柑子ブラコウジ.パイポパイポパイポのシューリンガングーリンダイ
のポンポコピーのポンポコナーの長久命の長助.

寿限無寿限無五劫の摺り切れ海砂利水魚の水行末雲来末風来末.食う寝る所に
住む所藪柑子ブラコウジ.パイポパイポパイポのシューリンガングーリンダイ
のポンポコピーのポンポコナーの長久命の長助.

寿限無寿限無五劫の摺り切れ海砂利水魚の水行末雲来末風来末.食う寝る所に
住む所藪柑子ブラコウジ.パイポパイポパイポのシューリンガングーリンダイ
のポンポコピーのポンポコナーの長久命の長助.

寿限無寿限無五劫の摺り切れ海砂利水魚の水行末雲来末風来末.食う寝る所に
住む所藪柑子ブラコウジ.パイポパイポパイポのシューリンガングーリンダイ
のポンポコピーのポンポコナーの長久命の長助.

\underline{寿限無寿限無五劫の摺り切れ}海砂利水魚の水行末雲来末風来末.食う寝る所に
住む所藪柑子ブラコウジ.パイポパイポパイポのシューリンガングーリンダイ
のポンポコピーのポンポコナーの長久命の長助.

=======================

\nlines{\Huge ■nlines これは1行 あああああああああああああああああああああ}

=======================

%\linespace{6}{\Huge ■linespace これは1行 ああああああああああああああああああ}


%\clearpage
\section{文字数・行数の確認}
\subsection{横27文字}
\noindent
1234567890
1234567890
1234567890

寿限無寿限無五劫の摺り切れ海砂利水魚の水行末雲来末風来末.食う寝る所に
住む所藪柑子ブラコウジ.パイポパイポパイポのシューリンガングーリンダイ
のポンポコピーのポンポコナーの長久命の長助.

\subsection{左右の段の行}

% 寿限無寿限無五劫の摺り切れ海砂利水魚の水行末雲来末風来末.食う寝る所に
% 住む所藪柑子ブラコウジ.パイポパイポパイポのシューリンガングーリンダイ
% のポンポコピーのポンポコナーの長久命の長助.

寿限無寿限無五劫の摺り切れ海砂利水魚の水行末雲来末風来末.食う寝る所に
住む所藪柑子ブラコウジ.パイポパイポパイポの\underline{シューリンガングーリンダイ}


\section{ははは}
ああああ

寿限無寿限無五劫の摺り切れ海砂利水魚の水行末雲来末風来末.食う寝る所に
住む所藪柑子ブラコウジ.パイポパイポパイポのシューリンガングーリンダイ
のポンポコピーのポンポコナーの長久命の長助.

寿限無寿限無五劫の摺り切れ海砂利水魚の水行末雲来末風来末.食う寝る所に
住む所藪柑子ブラコウジ.パイポパイポパイポのシューリンガングーリンダイ
のポンポコピーのポンポコナーの長久命の長助.

寿限無寿限無五劫の摺り切れ海砂利水魚の水行末雲来末風来末.食う寝る所に
住む所藪柑子ブラコウジ.パイポパイポパイポのシューリンガングーリンダイ
のポンポコピーのポンポコナーの長久命の長助.

寿限無寿限無五劫の摺り切れ海砂利水魚の水行末雲来末風来末.食う寝る所に
住む所藪柑子ブラコウジ.パイポパイポパイポのシューリンガングーリンダイ
のポンポコピーのポンポコナーの長久命の長助.

寿限無寿限無五劫の摺り切れ海砂利水魚の水行末雲来末風来末.食う寝る所に
住む所藪柑子ブラコウジ.パイポパイポパイポのシューリンガングーリンダイ
のポンポコピーのポンポコナーの長久命の長助.

ここまで\footnote{寿限無寿限無五劫の摺り切れ海砂利水魚の水
行末雲来末風来末.食う寝る所に住む所藪柑子ブラコウジ.パイポパイポパイ
ポのシューリンガングーリンダイのポンポコピーのポンポコナーの長久命の長
助.\underline{ここの下側が左右の段で揃うこと.}}.

\newpage
\noindent 1\par \noindent 2\par \noindent 3\par \noindent 4\par \noindent 5\par
\noindent 6\par \noindent 7\par \noindent 8\par \noindent 9\par \noindent 0\par
\noindent 11\par \noindent 2\par \noindent 3\par \noindent 4\par \noindent 5\par
\noindent 6\par \noindent 7\par \noindent 8\par \noindent 9\par \noindent 0\par
\noindent 21\par \noindent 2\par \noindent 3\par \noindent 4\par \noindent 5\par
\noindent 6\par \noindent 7\par \noindent 8\par \noindent 9\par \noindent 0\par
\noindent 31\par \noindent 2\par \noindent 3\par \noindent 4\par \noindent 5\par
\noindent 6\par \noindent 7\par \noindent 8\par \noindent 9\par \noindent 0\par
\noindent 41\par \noindent 2\par \noindent 3\par \noindent 4\par \noindent 5\par
\noindent 6\par \noindent 7\underline{□□□□□} %\par \noindent 8\par \noindent 9\par \noindent 0\par
%\noindent 1\par %\noindent 2\par \noindent 3\par \noindent 4\par \noindent 5\par

\end{document}