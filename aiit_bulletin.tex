% maketitleすると,footerの下部がずれてしまう現象について.
% 12Q ok, 13Q ok, 14Q ok
% 9pt NG, 10pt ok, 12pt ok, 13pt ok, 14pt NG, 17pt以上 未調査
\documentclass[a4j, 12Q, twocolumn, twoside]{jsarticle}
\usepackage{ajsarticle}

%%
%% 必要に応じてパッケージやマクロをここに追加してください
%%

\usepackage[dvipdfmx]{graphicx}		% 図の取り込み(PDF対応)用
\usepackage{okumacro}			% ルビ・圏点など
\usepackage{url}			% URLの出力

%%
%% 英語論文の場合は下記2行のコメントを外してください
%%

% \renewcommand{\tablename}{Table~}
% \renewcommand{\figurename}{Fig.~}

%%
%% タイトルを設定してください
%%

%% 和文表題
\title{産技大紀要フォーマットについて}
%% 和文著者名
\author{
  産技 太郎
  \thanks{産業技術大学院大学/Advanced Institute of Industrial
  Technology}
  産技 花子
  \thanks{公立大学法人首都大学東京/Tokyo Metropolitan University}
}
%% 英文表題
\etitle{Style and Layout of an AIIT Bulletin}
%% 英文著者名
\eauthor{
  Taro Sangi
  \thanksmark[1]
  Hanako Sangi
  \thanksmark[2]
}
%% 英文アブストラクト
\begin{abstract}
Lorem ipsum dolor sit amet, consectetur adipiscing elit. Donec aliquet
hendrerit dui at . Nunc blandit egestas felis non aliquet. Proin
malesuada dictum lacus eget elit accumsan, eu convallis urna
malesuada. Donec quis neque erat tempus congue ac eget est. Donec nec
dolor auctor, laoreet nisi id, tincidunt metus. Nunc eu scelerisque
nisi. Mauris vitae laoreet malesuada risus. Quisque at pharetra quam.
hendrerit augue sollicitudin vitae. Proin eget malesuada dictum
erat. In ullamcorper leo in volutpat bibendum hendrerit.(approx. 80
Words)
\end{abstract}
%% 英文キーワード
\keywords{AIIT, bulletin, 2013 (approx. 5 Keywords)}
%% 受領日
\receivedon{2013-09-30}

\begin{document}

% この下の\maketiteleにある[5pt]というオプションはタイトルと本文の間隔
% を調整するものです.
% 1ページ目の最下端の行が左右の段で揃わないとき,この値を0pt-6ptくらい
% の範囲で調整してください.
\maketitle[5pt]

\section{はじめに}
本稿では産業技術大学院大学紀要のフォーマットについて記す.

\section{執筆方法}
%\subsection{\LaTeX の場合}
\LaTeX のフォーマットを使う執筆者は,このファイルの中身を書き換えて使う
こと.
%
%\subsection{MS Wordの場合}
また,MS Wordを利用する場合は,スタイルを設定した
\texttt{.doc}ファイルを別添するのでそちらを用いること.

\section{原稿}
原稿は,日本語もしくは英語による完全版下(camera ready)原稿とする.
製版後の校正は原則として不可能であるため,誤字や脱字がないよう,特に念
を入れて仕上げる.刷り上がりは,6頁以上が望ましい.

\section{標題等ついて}
\subsection{標題}
標題は和文ならびに英文とする.英文原稿の場合は,和文表題を記述するとこ
ろに英文標題を記述し,通常の英文標題のところは削除すること.

\subsection{著者名・所属}
著者名も英文による原稿の場合は,通常の和文著者名のところに英文で記述し,
英文著者名のところは削除すること.所属も英文で記述すること.

\subsection{アブストラクト・キーワード}
和文ではなく英文で記述すること.アブストラクトは100語程度とし,キーワー
ドは5つ程度とする.

\subsection{標題等の割付}
見本に従って,[和文標題,和文著者名,英文標題,英文著者名,英文アブスト
ラクト,英文キーワード]及び[受領日,所属]の割付を行う.

\section{本文について}

\subsection{余白}
天地左右余白(マージン)・段間余白(コラムスペース)もこの見本に従う.
上下の余白には製本時にヘッダとページ番号を挿入するので,空白にしておく
こと.

\subsection{見出し}
原稿には,大見出し,中見出しなどを設け,それらを明瞭に区分する.さらに
細分を要するときは,著者の分類に委ねる.

\subsection{句読点}
句読点には,全角ピリオド(.),全角コンマ(,)を用いること.


\section{参考文献について}
参考文献は,通し番号とし,本文中では,当該事項または人名などの参考とす
る後に,\cite{TRA96HuHaCo},\cite{SICE02Yo}--\cite{Asa02Ar} のように記
す.文章の末尾に記す必要がある場合には,句読点の前に記す.

参考文献は,原則として,雑誌の場合は,著者,標題,雑誌名,巻,号,頁,
年の順に記す.また,著書の場合は,著者,書名,発行所,発行年の順に記す.
参考文献例を本文の最後に挙げるので参考されたい.

\section{図・表について}
\subsection{図表のキャプション}
図・表には,図1,図2,表1,表2 のように論文全体で通し番号をつけること.
英文の場合には,Fig. 1,Fig. 2,Table 1,Table 2のように,番号をつける
こと.通し番号,標題は本文と同じ書体を使用すること.表のキャプションは
表の上に,図のキャプションは図の下につけること.

\begin{figure}
 \begin{center}
  \includegraphics[width=0.8\linewidth]{aiit_symbol.eps}
 \end{center}
 \caption{図の説明}
 \label{fig:one}
\end{figure}

\begin{table}
  \caption{表のキャプション}
  \begin{center}
    \begin{tabular}{|c|c|c|} \hline
    A & B & C \\ \hline
    $A_{1}$ & $B_{1}$ & $C_{1}$ \\ \hline
    $A_{2}$ & $B_{2}$ & $C_{2}$ \\ \hline
    $A_{3}$ & $B_{3}$ & $C_{3}$ \\ \hline
    $A_{4}$ & $B_{4}$ & $C_{4}$ \\ \hline
    \end{tabular}
  \end{center}
\end{table}

\subsection{図表に関する注意}
図・表は,印刷に十分耐えうるものでなければならない.刷り上がり時の文字
が小さすぎないよう十二分に配慮し,線の太さにも注意する.

図・表に色刷りを必要とする場合は,別途連絡すること.ただし,製本上の都
合で色刷り頁を設けることができない場合もありうる\footnote{テスト}.

\section{おわりに}
本稿では産業技術大学院大学紀要のフォーマットについて記した.

\begin{thebibliography}{99}
\bibitem{TRA96HuHaCo} 
S. Hutchinson, G. D. Hager and P. I. Corke,
``A Tutorial on Visual Servo Control,''
{\it IEEE Trans. Robotics and Automation},
Vol.~12, No.~5, pp.~651--670, 1996.
%
\bibitem{SICE02Yo}
吉川恒夫,
``ロボット技術,''
計測と制御, Vol.~41, No.~1, pp.~17--21, 2002.
%
\bibitem{Joh06SpHuVi} 
M. W. Spong, S. Hutchinson and M. Vidyasagar, 
{\it Robot Modeling and Control}, 
John Wiley \& Sons, 2006. 
%
\bibitem{Asa02Ar}
有本 卓, 新版 ロボットの力学と制御, 朝倉書店, 2002. 
\end{thebibliography}

\end{document}