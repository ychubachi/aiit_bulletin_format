%#BIBTEX pbibtex aiit_bulletin
%% 産業技術大学院大学紀要フォーマット
%% Created: 2013-06-30 Y.Chubachi
\documentclass[a4j, 12Q, twocolumn, twoside]{jsarticle}
\usepackage{aiit_bulletin}		% 紀要のスタイル

%%
%% 1.必要に応じてパッケージやマクロをここに追加
%%
\usepackage{newcent}			% Centuryフォント
\usepackage[dvipdfmx]{graphicx}		% 図の取り込み(PDF対応)用
\usepackage{okumacro}			% ルビ・圏点など
\usepackage{url}			% URLの出力

%%
%% 2.英語論文の場合は下記2行のコメントを外してください
%%

% \renewcommand{\tablename}{Table~}
% \renewcommand{\figurename}{Fig.~}

%%
%% 3.タイトルを設定してください
%%
%% 本文が英文の場合は,\etitle, \eauthorを削除し,\titleと\autherに
%% 英文を記入してください.
%%

%%
%% 和文表題
%% 
\title{産業技術大学院大学紀要の書式について\\-- 2013年度(\LaTeX)版 --}

%%
%% 和文著者名
%%
%% \thanksの中で改行する場合\\ではなく\newlineを使用する
%%
\author{
  産技 太郎
  \thanks{産業技術大学院大学 \newline
  Advanced Institute of Industrial Technology}
  産技 花子
  \thanks{公立大学法人首都大学東京 \newline
  Tokyo Metropolitan University}
}

%%
%% 英文表題
%%
\etitle{Style and Layout of an AIIT Bulletin\\-- 2013 (\LaTeX) --}

%%
%% 英文著者名
%%
\eauthor{
  Taro Sangi
  \thanksmark[1]
  Hanako Sangi
  \thanksmark[2]
}

%%
%% 英文アブストラクト
%%
\begin{abstract}
Lorem ipsum dolor sit amet, consectetur adipiscing elit. Donec aliquet
hendrerit dui at. Nunc blandit egestas felis non aliquet. Proin
malesuada dictum lacus eget elit accumsan, eu convallis urna
malesuada. Donec quis neque erat tempus congue ac eget est. Donec nec
dolor auctor, laoreet nisi id, tincidunt metus. Nunc eu scelerisque
nisi. Mauris vitae laoreet malesuada risus. Quisque at pharetra quam.
hendrerit augue sollicitudin vitae. Proin eget malesuada dictum
erat. In ullamcorper leo in volutpat bibendum hendrerit.(approx. 80
Words)
\end{abstract}

%%
%% 英文キーワード
%%
\keywords{AIIT, bulletin, 2013 (approx. 5 Keywords)}

%%
%% 受領日
%%
\receivedon{2013-09-30}

\begin{document}
%%
%% 4.タイトルを出力
%%
%% \maketiteleにあるオプション[0pt]はタイトルと本文の間隔
%% を調整するものです.
%% 1ページ目の脚注の下端と本文の下端が左右の段で揃わない
%% とき,この値を0pt〜8ptくらいの範囲で調整してください.
\amaketitle[-1pt]

%%
%% 5.本文
%%
\section{はじめに}
本稿では産業技術大学院大学紀要の書式について記す.

\section{原稿}
\subsection{カメラレディ原稿}
原稿は,日本語もしくは英語による完全版下(camera ready)原稿とする.
製版後の校正は原則として不可能であるため,誤字や脱字がないよう,特に念
を入れて仕上げる.刷り上がりは,6頁以上が望ましい.

\subsection{余白}
天地左右余白(マージン)・段間余白(コラムスペース)はこの見本に従う.
上下の余白には製本時にヘッダとページ番号を挿入するので,空白にしておく
こと.

\section{標題について}
\subsection{標題}
標題は和文ならびに英文とする.英文原稿の場合は,和文表題を記述する箇所
に英文標題を記述し,英文標題の箇所は削除すること.

\subsection{アブストラクト・キーワード}
和文ではなく英文で記述すること.アブストラクトは80語程度とし,キーワー
ドは5つ程度とする.

\subsection{標題等の割付}
見本に従って,[和文標題,和文著者名,英文標題,英文著者名,英文アブスト
ラクト,英文キーワード]及び[受領日,所属]の割付を行う.

\subsection{英文原稿の場合}
英文による原稿の場合は,和文著者名のところに英文で記述し,英文著者名の
ところは削除すること.所属も英文で記述すること.

\section{本文について}

\subsection{句読点}
和文の句読点には全角ピリオド(.),全角コンマ(,)を用いること.

\subsection{見出し}
原稿には,大見出し,中見出しなどを設け,それらを明瞭に区分する.さらに
細分を要するときは著者に委ねる.

\section{参考文献について}
参考文献は,通し番号とし,本文中では,当該事項または人名などの参考とす
る後に,\cite{okumura},あるいは,\cite{takeuchi1986new,sutherland2011scrum} のよ
うに記す.文章の末尾に記す必要がある場合には,句読
点の前に記す\cite{IT人材白書2012}.

参考文献は,原則として,雑誌の場合は,著者,標題,雑誌名,巻,号,頁,
年の順に記す.また,書籍の場合は,著者,書名,発行所,発行年の順に記す.
参考文献例を本文の最後に挙げるので参考されたい.

\section{脚注等について}
\subsection{脚注}
脚注は段組の下部に記載する\footnote{脚注の例}.

\subsection{ルビ・圏点}
\ruby{難読漢字}{なんどくかんじ}にはルビを振ることができる.また,強調
したい箇所には\kenten{圏点をつける}ことができる.

\section{図・表について}
\subsection{図表のキャプション}
図・表には,図1,図2,表1,表2 のように論文全体で通し番号をつけること.
英文の場合には,Fig. 1,Fig. 2,Table 1,Table 2のように,番号をつける
こと.通し番号,標題は本文と同じ書体を使用すること.

表のキャプションは表の上に,図のキャプションは図の下につけること.

\begin{figure}
 \centering	% \begin{center} は使わない.余計な余白が入る.
 % 図を9行取り(9\baselineskip)して挿入する例
 \includegraphics[height=9\baselineskip]{aiit_symbol.eps}
 \caption{図のキャプション}
 \label{fig:one}
\end{figure}

\subsection{図表に関する注意}
図・表は,印刷に十分耐えうるものでなければならない.刷り上がり時の文字
が小さすぎないよう十二分に配慮し,線の太さにも注意する.

図・表に色刷りを必要とする場合は,別途連絡すること.ただし,製本上の都
合で色刷り頁を設けることができない場合もありうる.

\begin{table}
 \centering
 \caption{表のキャプション}
 \begin{tabular}{c|ccc} \hline \hline
   & A & B & C \\ \hline
   R1 & $A_{1}$ & $B_{1}$ & $C_{1}$ \\
   R2 & $A_{2}$ & $B_{2}$ & $C_{2}$ \\
   R3 & $A_{3}$ & $B_{3}$ & $C_{3}$ \\
   R4 & $A_{4}$ & $B_{4}$ & $C_{4}$ \\ \hline
 \end{tabular}
 \vskip 10pt % 左右の段の行が揃うように調整
\end{table}

\section{テンプレート}
\LaTeX を利用する執筆者は,\texttt{aiit\_bulletin.tex}を書き換えて使う
こと.また,MS Wordを利用する場合は,\texttt{aiit\_bulletin.docx}ファイ
ルを用いること.

\section{おわりに}
本稿では産業技術大学院大学紀要のフォーマットについて記した.

\bibliographystyle{junsrt}
\bibliography{bibliography}

\appendix
\section{注意事項}
\subsection{フォントの準備}
LaTeXとWordで同じ字体に揃えるため,和文には「IPAexフォント(2書体パッ
ク)」を用いる.次のサイトからインストールすることができる.

\url{http://ipafont.ipa.go.jp/ipaexfont/download.html}

なお,TeX Live 2012 以降およびW32TeX [2013/06/06] 以降の
luatexja.tar.xzに は同梱されている.

\subsection{版面と余白}
何らかの事情で書式を自分で設定する場合,本文の書式は次の通りとする.

\begin{itemize}
 \item フォントの大きさは8.5pt(=12Q) とする.
 \item 2段組とし1行27文字で,段間は2文字分あける.
 \item 行間は字の高さの75\%とし,1頁47行とする.
 \item 余白は天(上部)30mm,ノド(内側)25mmとする.
\end{itemize}

これら以外の書式は可能な限りこの見本に合わせる.

\end{document}